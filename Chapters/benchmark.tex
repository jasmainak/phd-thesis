% Chapter 4

\chapter{Benchmarking on Phantom datasets}
\label{chapter:benchmark}
\noindent\makebox[\linewidth]{\rule{0.75\paperwidth}{0.4pt}}
\noindent\makebox[\linewidth]{\rule{0.75\paperwidth}{0.4pt}}

\localtableofcontents % local toc

\noindent\makebox[\linewidth]{\rule{0.75\paperwidth}{0.4pt}}
\noindent\makebox[\linewidth]{\rule{0.75\paperwidth}{0.4pt}}
\newpage

\section{Introduction}
\section{Sphere models}
The most commonly used head model assumes that it is made up of a set of nested concentric spheres, each with homogeneous and isotropic conductivity. Under this assumption, both the EEG and MEG problems admit to wellknown closed form solutions. Here, we describe the forward solutions for both problems for the spherical model using kernel matrices. The matrices are explicitly stated here in Cartesian coordinates. While these results are well known, presentation in this form is novel and suitable for direct use in inverse methods with unconstrained dipole orientations.
~\cite{mosher1999eeg}
\section{Results}
\subsection{dipole fitting}
\subsection{Gamma Map}
\subsection{RAP MUSIC}
\subsection{MxNE}
\subsection{irMxNE}
\subsection{TF-MxNE}
\subsection{irTF-MxNE}
\subsection{Critical comparison of these M/EEG source localization}
\subsection{Discussion \& Conclusion}