% Chapter 5

\chapter{Conclusion \& Perspectives}
\label{chapter:conclusion}

This thesis demonstrated various ways to solve the MEG/EEG source localization problem. It tackles specific challenges faced on actual state-of-the-art techniques, and tries to improve them point by point:
\begin{itemize}
\item Promoting structured sparsity in the TF domain has been proved to applicable for reconstructing non-stationary sources, although it needs to fix some parameters related to the Gabor transform, which are involved in the TF resolution. The first improvement of this thesis was to tackle the choice of these parameters, which can be very detrimental for the analysis of brain waves with variable TF characteristics. It gives a new technique based on a multi-scale TF mixed norm allowing us to localize more accurately in space and time the source estimates (see Chapter~\ref{chapter:multiscale}).

\item The formulation of the MEG/EEG inverse problem has been mostly written as a penalized regression way, meaning it needs to introduce a prior knowledge as a regularization term to the objective function. This results to add a hyperparameter to the model which needs to be tuned. This thesis tackles this second challenge with two ways, both reformulating the problem as done in the Bayesian community. The Bayesian formulation allows to hierarchically add hyperparameters to be alternatively estimated with the main parameters of the model (the sources). The two main advantages are: the direct estimation of the hyperparameter, and the ability to use sampling to investigate the uncertainty of these solvers. These two points were presented in Chapter~\ref{chapter:bayesian}.

\item An important step after each new techniques developed is to validate them with a comparison to the other exisiting methods. This has been for a while a hard step as it is always hard to develop a good and a realistic simulations where to apply the new techniques. For this aim, several study has been investigating phantom dataset, which is a dataset of a mimic phantom head replicating a realistic environment. Chapter~\ref{chapter:benchmark} shows a comparison of the solvers presented in this thesis on phantom datasets.
\end{itemize}

This thesis was based on a long research line started my supervisor Alexandre Gramfort, and then his former PhD Student Daniel Strohmeier, and was meant to improve several parts of the MEG/EEG source localization issues encountered so far. The points cited before were mainly the ones developed in this thesis, however several non-trivial ones remain open questions for further work to make best use of the available neuroimaging data:
\begin{itemize}
\item Although we found a way to solve the problem of source localization in the TF domain when having a mixture of brain signal in the data, it is still a non-trivial task to set the parameters of the multi-scale dictionary. As presented before, another possible way is to learn models that are good enough to capture the rich frequency content, and the morphology of the brain signal. The dictionary learning research line has given pretty nice results so far on electrophisiological signals~\cite{jas2017learning,jost2006motif,brockmeier2016learning,hitziger2017adaptive}, which makes the technique completely autonomous and a data-driven approach.

\item The spatio-temporal techniques presented in this thesis are designed for the analysis of averaged evoked responses to ensure a descent SNR. Future work can be directed on how to make these techniques applicable on single trial data. An idea is to localize each trial separately by imposing an additional constraint onto the model such as the active set must be consistent over all trials~\cite{strohmeier2012meg,strohmeier2012biomag}.

\item While invoking new constraints onto the model, another line of future work can be on the optimization side of solving the MEG/EEG inverse problem. In machine learning, various papers have been investigating mathematical and computational challenges to better tackle the inverse problem in general. One possible direction for the MEG/EEG inverse problem to take is in order to improve the computational complexity, where the proposed approaches need to be competitive in terms of running time. This results in research such as how to fasten the algorithms, an practical example which was used in this thesis is the use of an active set. A more sophisticated approach would be to apply screening rules, \textit{i.e.}, find in advance the involved sources to compute the solution only for them and avoid spending time on computing sources which will be inactive at the end~\cite{fercoq-etal:2015,massias2017safe,massiasgap,ndiaye2016gap,ndiaye2017efficient}.

\item The proposed method in the TF domain still lack an automatic model selection criterion to set the two regularization hyperparameters (one over space, second over time). Chapter~\ref{chapter:bayesian} presented a way to automatically set the hyperparameter for the mixed norm (MxNE) approach, which has only one regularization parameter over space. A further work can rewrite the problem for TF-MxNE, or investigate other model selection criterion .

\item The novel and some of state-of-the-art approaches have been tested using three phantom datasets. A future work would be to investigate more this validation to compare their capabilities with a more sophisticated data, \textit{i.e.}, two or more dipoles in a row, instead of only one dipole as presented here.
\end{itemize} 
